\documentclass{beamer}
\mode<presentation>

\usepackage{pgfpages}
\usepackage{bbm}
\usepackage{amsmath}
\usepackage{amssymb} 
\usepackage{amsthm}
\usepackage{epsfig}
\usepackage{graphicx}

\graphicspath{{./figures/}}

\makeatletter 
\def\newblock{\beamer@newblock} 
\makeatother  

% set color and font of frametitles
\setbeamercolor{frametitle}{fg=blue}
%\setbeamerfont{frametitle}{series=\bfseries}

\newcommand{\red}[1]{{\color{red} #1}}
\newcommand{\blue}[1]{{\color{blue} #1}}
\newcommand{\avg}[1]{\left\langle #1 \right\rangle}

\newcommand{\code}[1]{{\fontfamily{pcr}\selectfont \textbf{#1}}}

%%%%%%%%%%%%%%%%%%%%%%%%%%%%%%%%%%%%%%%%%%%%%%%%%%%%%%%%%%%%%%%%%
%% SETTINGS

% set item type and color
\setbeamercolor{item}{fg=black}  % set color
\setbeamertemplate{itemize items}[circle] % if you want a circle
\setbeamertemplate{itemize subitem}[ball] % if you wnat a ball
\setbeamertemplate{itemize subsubitem}[triangle] % if you want a triangle

% set spacing for itemize
\setbeamertemplate{itemize/enumerate subbody begin}{\vspace{0.1cm}}
\setbeamertemplate{itemize/enumerate subbody end}{\vspace{0.1cm}}

%% END SETTINGS
%%%%%%%%%%%%%%%%%%%%%%%%%%%%%%%%%%%%%%%%%%%%%%%%%%%%%%%%%%%%%%%%%%%

% remove navigation symbols
\beamertemplatenavigationsymbolsempty 

\title{Version control management and research collaboration using git and github}
\subtitle{An introduction}
\author{APSIS group}
\institute{MCC Berlin}
\date{July 11th, 2019}

\AtBeginSection[]
  {
     \begin{frame}<beamer>
     \frametitle{}
     \tableofcontents[currentsection]
     \end{frame}
  }

\begin{document}

\maketitle


\section{What is git and GitHub?}

% ==============================================================
\begin{frame}
\frametitle{What is version control management?}

Software to keep track of the history and different versions of files within project folders

\end{frame}
%--------------------------------------------------------------


% ==============================================================
\begin{frame}
\frametitle{What is git?}

\begin{itemize}
\item git is a program for version control

\item designed for distributed software development

\item created by Linus Torvalds for his work on the Linux kernel
\end{itemize}

Explain idea of a git repository

\end{frame}
%--------------------------------------------------------------

\begin{frame}
\frametitle{What is GitHub?}

Explain the idea of a remote repository

Explain github (and providers of remote repositories like gitlab, bitbucket, SourceForge, Launchpad ...)

\end{frame}


\section{Why should I use it?}

\begin{frame}
\frametitle{Why use version control in research?}

\begin{itemize}
\item getting some order in the mess
\begin{itemize}
\item data
\item software code/scripts
\item manuscripts for papers 
\end{itemize}

\item sharing your code or 

\item collaboration and attribution of work
\end{itemize}

\end{frame}
%--------------------------------------------------------------

\section{How can I use it?}

% ==============================================================
\begin{frame}
\frametitle{Starting a repository}

Command line vs. GUIs (gitg, SourceTree, GitHub Desktop etc.)

\end{frame}
%--------------------------------------------------------------


% ==============================================================
\begin{frame}
\frametitle{Starting a repository}


\code{git init}

\code{git clone}

\end{frame}
%--------------------------------------------------------------

% ==============================================================
\begin{frame}
\frametitle{Managing the repository}

\code{git status}

\code{git diff}

staging:

\code{git add [--all]}

\code{git commit [-a]}

\end{frame}
%--------------------------------------------------------------

% ==============================================================
\begin{frame}
\frametitle{Branches and merges}

explain what branches are

\code{git branch}

\code{git merge \emph{branchname}}

\code{git checkout}

\end{frame}
%--------------------------------------------------------------

% ==============================================================
\begin{frame}
\frametitle{Interacting with remote repositories}

\code{git pull}

\code{git push}

Warning: careful with copyrighted materials in public repositories

forking and pull request for working on repository for which you are no collaborator

\end{frame}
%--------------------------------------------------------------

% ==============================================================
\begin{frame}
\frametitle{Further useful commands and tools}

.gitignore file

create doi for citations:
https://guides.github.com/activities/citable-code/

\end{frame}
%--------------------------------------------------------------

% ==============================================================
\begin{frame}
\frametitle{}

{\huge
Questions?
}

\end{frame}
%--------------------------------------------------------------

% ==============================================================
\begin{frame}
\frametitle{Practice / task}

\begin{itemize}
\item clone remote repository with

\code{git clone https://github.com/mcc-apsis/git-intro.git}

\item add some question or feedback to the presentation in the file

\item add and commit changes

\item pull changes already made by other

\item push your own changes
\end{itemize}




\end{frame}
%--------------------------------------------------------------


\end{document}


